\documentclass[12pt, a4paper]{book}

\usepackage[utf8]{inputenc}
\usepackage[slovene]{babel}
\usepackage[T1]{fontenc}
\usepackage{lmodern}

\usepackage{graphicx}
\usepackage{float}

\usepackage{amssymb}
\usepackage{amsmath,amsfonts}
\usepackage{amsthm}
\usepackage{mathtools}
\usepackage{subfigure}
\usepackage{wrapfig}
\usepackage{lipsum}
\usepackage{tikz}

\usepackage{array}
\usepackage{multicol}

\usepackage{booktabs}

\usepackage[
font=small,
format=plain,
labelfont=bf,
textfont=it,
justification=centerlast
]{caption}

\usepackage{enumerate}

\newcounter{pcount}

\title{Pravila lepega risanja grafov}
\author{Nik Erzetič}
\date{}
    
\begin{document}
	\maketitle
    
    \tableofcontents
    \newpage
    
    \chapter{Osnovna pravila lepega risanja grafov}
    \begin{list}{\textbf{\arabic{pcount}. pravilo lepega risanja grafov:}}{\usecounter{pcount}}
    	\item 
    	Vprašaj se, ali je graf sploh potrerbno narisati. Odgovor bo neodvisno od situacije pritrdilen - pomembno je le, da vedno podvomiš v smiselnost svojih dejanj.
        \item
        Seznani se z risalno površino in razpoložljivim orodjem.
        \item
        Za uporabo je najboljši karo zvezek, druga najboljša izbira je brezčrtan. Črtast zvezek je spremenljiv le v primeru, da si naletel na za človeštvo primerno odkritje in nimaš pri roki drugega papirja. Visoki karo je, preprosto rečeno, abominacija, veliki karo pa spada v prvi razred osnovne šole.
        \item
        Ne uporabljaj ravnila, če od tega ni odvisno tvoje ali, v posebnem primeru, koga drugega življenje.
        \item
        Za risanje koordinatnih osi uporabljaj svinčnik. Uporaba nalivnega peresa ali kemičnega svinčnika je sprejemljiva le v primeru prerisovanja grafa  s table ali ob simbolni ponazoritvi grafa.
    \end{list}
\end{document}








